\documentclass{article}
\usepackage{graphicx} % Required for inserting images
\usepackage{listings} % For code formatting
\usepackage{xcolor} % For coloring code

\title{OOP - Array di Oggetti}
\author{SIAROU SASHA}
\date{January 2025}

\lstset{
    language=Java, % Change to C++ if needed
    basicstyle=\ttfamily\small,
    keywordstyle=\color{blue}\bfseries,
    commentstyle=\color{gray}\itshape,
    stringstyle=\color{red},
    numbers=left,
    numberstyle=\tiny\color{gray},
    frame=single,
    breaklines=true
}

\begin{document}

\maketitle

\section{Introduction}
In programmazione orientata agli oggetti (OOP), un array di oggetti è una struttura dati che permette di memorizzare e gestire più istanze di una classe. Questo è utile per lavorare con un insieme di oggetti correlati.

\section{Esempio di Array di Oggetti}
Di seguito è riportato un esempio in Java che mostra come creare e utilizzare un array di oggetti:

\subsection{Definizione della Classe}
\begin{lstlisting}
// Classe Persona con attributi nome e età
class Persona {
    String nome;
    int eta;

    // Costruttore
    public Persona(String nome, int eta) {
        this.nome = nome;
        this.eta = eta;
    }

    // Metodo per stampare le informazioni
    public void stampaInfo() {
        System.out.println("Nome: " + nome + ", Età: " + eta);
    }
}
\end{lstlisting}



\subsection{Utilizzo di un Array di Oggetti}
\begin{lstlisting}
public class Main {
    public static void main(String[] args) {
        // Creazione di un array di oggetti Persona
        Persona[] persone = new Persona[3];

        // Inizializzazione degli oggetti nell'array
        persone[0] = new Persona("Alice", 25);
        persone[1] = new Persona("Bob", 30);
        persone[2] = new Persona("Charlie", 35);

        // Accesso e utilizzo degli oggetti
        for (Persona persona : persone) {
            persona.stampaInfo();
        }
    }
}
\end{lstlisting}

\section{Conclusione}
Un array di oggetti è una tecnica potente per gestire collezioni di istanze di classe. L'esempio sopra mostra come definire una classe, creare un array di oggetti e iterare attraverso di esso.

\end{document}
